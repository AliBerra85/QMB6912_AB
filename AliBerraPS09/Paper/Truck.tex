\documentclass[11pt]{paper}
\usepackage{palatino}
\usepackage{amsfonts,amsmath,amssymb}
% \usepackage{graphicx}

\usepackage{listings}
\usepackage{textcomp}
\usepackage{color}

\definecolor{dkgreen}{rgb}{0,0.6,0}
\definecolor{gray}{rgb}{0.5,0.5,0.5}
\definecolor{mauve}{rgb}{0.58,0,0.82}

\lstset{frame=tb,
  language=R,
  aboveskip=3mm,
  belowskip=3mm,
  showstringspaces=false,
  columns=flexible,
  basicstyle={\small\ttfamily},
  numbers=none,
  numberstyle=\tiny\color{gray},
  keywordstyle=\color{blue},
  commentstyle=\color{dkgreen},
  stringstyle=\color{mauve},
  breaklines=true,
  breakatwhitespace=true,
  tabsize=3
}



\ifx\pdftexversion\undefined
    \usepackage[dvips]{graphicx}
\else
    \usepackage[pdftex]{graphicx}
    \usepackage{epstopdf}
    \epstopdfsetup{suffix=}
\fi

\usepackage{subfig}


% This allows pdflatex to print the curly quotes in the
% significance codes in the output of the GAM.
\UseRawInputEncoding

\begin{document}

%%%%%%%%%%%%%%%%%%%%%%%%%%%%%%%%%%%%%%%%
% Problem Set 7
%%%%%%%%%%%%%%%%%%%%%%%%%%%%%%%%%%%%%%%%

\pagestyle{empty}
{\noindent\bf Spring 2021 \hfill Ali Berra, Alex Georgiopoulus, Caroline Gear}
\vskip 16pt
\centerline{\bf University of Central Florida}
\centerline{\bf College of Business}
\vskip 16pt
\centerline{\bf QMB 6911}
\centerline{\bf Capstone Project in Business Analytics}
\vskip 10pt
\centerline{\bf Solutions:  Problem Set \#8}
\vskip 32pt
\noindent
% 
\section{Data Description}

This analysis follows the script \texttt{Trucks\_Reg\_Model.R} to produce a more accurate model for used trucks cost with the data from \texttt{UsedTrucks.dat} in the \texttt{Data} folder. 
The dataset includes the following variables.
\begin{table}[h!]
\begin{tabular}{l l l}

$type_i$ & = & sale type \\

$pauc$ & = & price when sold at auction \\
$ pret$ & = &price when sold retail \\ 
$mileage$ & = & odometer \\ %, $0$ otherwise \\
$make$ & = &make of vehicule \\
$year$ & = &model year of vehicle \\ 
$damage$ & = & an index of damage to vehicle, 1 little damage, 10 a lot\\
$dealer$ & = & dealer id \\ 
$ror$& = &rate-of-return\\
$cost$& = &net amount given to trade in\\


\end{tabular}
\end{table}
%
From problem set 7, we build a model that can predict the cost of the vehicule depending 
on the damage of the trucks. we will add quadratic specification for damage 
which will increase the relationship between cost and damage. 



%%%%%%%%%%%%%%%%%%%%%%%%%%%%%%%%%%%%%%%%
\clearpage
\section{Linear Regression Model}
%%%%%%%%%%%%%%%%%%%%%%%%%%%%%%%%%%%%%%%%

A natural staring point is the recommended linear model
from Problem Set \#7. 

\subsection{Quadratic Specification for Damage}

we can say that the more a Truck is damaged the less likely it will sell. 
to impliment this in the model we squared the damage variable to have more impact inthe realtionship.


The results of this regression specification are shown in 
Table \ref{tab:reg_sq_damage}. 
% 

\begin{table}
\begin{center}
\begin{tabular}{l c}
\hline
 & Model 1 \\
\hline
(Intercept)     & $8.76488^{***}$  \\
                & $(0.08135)$      \\
squared\_damage & $0.00459^{*}$    \\
                & $(0.00214)$      \\
make            & $-0.00619$       \\
                & $(0.00537)$      \\
damage          & $-0.04677$       \\
                & $(0.02543)$      \\
dealer          & $0.00196$        \\
                & $(0.00417)$      \\
mileage         & $-0.00001^{***}$ \\
                & $(0.00000)$      \\
age             & $-0.07163^{***}$ \\
                & $(0.01114)$      \\
type            & $0.04438^{*}$    \\
                & $(0.02182)$      \\
\hline
R$^2$           & $0.19087$        \\
Adj. R$^2$      & $0.19029$        \\
Num. obs.       & $9861$           \\
\hline
\multicolumn{2}{l}{\scriptsize{$^{***}p<0.001$; $^{**}p<0.01$; $^{*}p<0.05$}}
\end{tabular}
\caption{Quadratic Model for Trucks cost}
\label{tab:reg_sq_damage}
\end{center}
\end{table}

% 
according to our model milage and age are more significant predictor than damge with -5.053e-06 and -7.163e-02 coeffiecient respectively. 
the P value for damage 0.06 and for squared damage 0.03 which is slightly significant when squared







%%%%%%%%%%%%%%%%%%%%%%%%%%%%%%%%%%%%%%%%
\clearpage
\section{Nonlinear Specifications}
%%%%%%%%%%%%%%%%%%%%%%%%%%%%%%%%%%%%%%%%


% \clearpage
\subsection{Nonparametric Specification for Horsepower}


The specification in 
Table \ref{tab:reg_sq_damage}
assumes a quadratic functional form for
the relationship between cost and damage. 
To consider the damage variable alone, 
while accounting for the effects of other variables, 
one can fit a nonparametric model to the residuals 
from a model of Trucks cost, 
after regressing Trucks cost on the other variables. 
This leaves only the variation in trucks cost that is not explained by the other variables. 
Going one step further, perform the same transformation to the damage variable:
take the residuals from a model of damage, 
after regressing damage on the other variables. 
This allows a model that would fit exactly the same as if it were estimated within a full model with all variables included. 

The models shown in
Table \ref{tab:reg_sq_cost_fwl}
illustrate this possibility. 
Model 1 is the original model in 
Table \ref{tab:reg_sq_damage}. 
Model 2 is a regression omitting the damage variables. 
Model 3 is a regression to predict damage with the other explanatory variables in Model 2.
Finally, Model 4 shows the coefficients for damage
from a regression of the residuals of Model 2
on the residuals from Model 3. 
Notice that these coefficients match those in Model 1. 
You might notice a slight difference in the standard errors, however, 
because these are calculated assuming coefficients 
for two variables, damage and squared damage,
rather than the full suite of ten parameters.
This equivalence of the coefficients can be used to fit
nonlinear models between a pair of variables by 
partialing out the effect of the other variables, 
using a mathematical result called the Frisch-Waugh-Lovell (FWL) theorem, 
maned after early statisticians and econometricians who used these methods. 


\begin{table}
\begin{center}
\begin{tabular}{l c c c c c}
\hline
 & Model 1 & Model 2 & Model 3 & Model 4 & Model 5 \\
\hline
(Intercept)      & $8.76488^{***}$  & $8.65101^{***}$  & $2.83163^{***}$ & $4.04282^{***}$ &               \\
                 & $(0.08135)$      & $(0.05049)$      & $(0.06877)$     & $(0.81893)$     &               \\
squared\_damage  & $0.00459^{*}$    &                  &                 &                 &               \\
                 & $(0.00214)$      &                  &                 &                 &               \\
make             & $-0.00619$       & $-0.00621$       & $-0.00411$      & $-0.04520$      &               \\
                 & $(0.00537)$      & $(0.00537)$      & $(0.00731)$     & $(0.08708)$     &               \\
damage           & $-0.04677$       &                  &                 &                 &               \\
                 & $(0.02543)$      &                  &                 &                 &               \\
dealer           & $0.00196$        & $0.00207$        & $0.00528$       & $0.07682$       &               \\
                 & $(0.00417)$      & $(0.00417)$      & $(0.00568)$     & $(0.06761)$     &               \\
mileage          & $-0.00001^{***}$ & $-0.00001^{***}$ & $0.00000$       & $0.00001$       &               \\
                 & $(0.00000)$      & $(0.00000)$      & $(0.00000)$     & $(0.00002)$     &               \\
age              & $-0.07163^{***}$ & $-0.06985^{***}$ & $0.32609^{***}$ & $3.70576^{***}$ &               \\
                 & $(0.01114)$      & $(0.01088)$      & $(0.01482)$     & $(0.17644)$     &               \\
type             & $0.04438^{*}$    & $0.04590^{*}$    & $0.34444^{***}$ & $3.83624^{***}$ &               \\
                 & $(0.02182)$      & $(0.02168)$      & $(0.02952)$     & $(0.35157)$     &               \\
damage\_resid    &                  &                  &                 &                 & $-0.04677$    \\
                 &                  &                  &                 &                 & $(0.02542)$   \\
damage\_2\_resid &                  &                  &                 &                 & $0.00459^{*}$ \\
                 &                  &                  &                 &                 & $(0.00213)$   \\
\hline
R$^2$            & $0.19087$        & $0.19044$        & $0.50020$       & $0.47707$       & $0.00053$     \\
Adj. R$^2$       & $0.19029$        & $0.19003$        & $0.49994$       & $0.47680$       & $0.00032$     \\
Num. obs.        & $9861$           & $9861$           & $9861$          & $9861$          & $9861$        \\
\hline
\multicolumn{6}{l}{\scriptsize{$^{***}p<0.001$; $^{**}p<0.01$; $^{*}p<0.05$}}
\end{tabular}
\caption{Quadratic Model for Trucks cost: FWL Regressions}
\label{tab:reg_sq_damage_fwl}
\end{center}
\end{table}


\pagebreak 
To illustrate the fit of the model, 
Figure \ref{fig:dev_vs_cost} shows a scatter plot 
of the residual log cost on damage. 
The observations are shown in blue
and the fitted values are shown in red.
The variation in the fitted values results from the 
fact that it is not plotted against the transformed excess damage variable used in the regressions.
Still, the quadratic pattern is apparent
and appears to match the data. 

\begin{figure}[h!]
  \centering
  \includegraphics[scale = 0.5, keepaspectratio=true]{../Figures/dev_vs_cost}
  \caption{Linear-Quadratic Model for Truck cost} \label{fig:dev_vs_cost}
\end{figure}



\pagebreak
As a comparison, Figure \ref{fig:dev_vs_damage_dev} 
augments the above by showing the plot against the 
residuals from the regression for damage:
the ``excess damage'' compared to what would be 
expected given the other characteristics of a truck. 



\begin{figure}[h!]
  \centering
  \includegraphics[scale = 0.5, keepaspectratio=true]{../Figures/dev_vs_cost_dev}
  \caption{Linear-Quadratic Model for Tractor Prices: damage} \label{fig:dev_vs_cost_dev}
\end{figure}

\clearpage
Now consider a nonparametric specification for 
the relationship between cost and damage.
Figure \ref{fig:dev_np_vs_damage_dev} 
overlays the nonparametric estimate (shown in green) with the above in 
Figure \ref{fig:dev_vs_damage_dev}.
The pattern has more variation in slope but 
closely follows the prediction from the quadratic model. 
So far, it appears that the quadratic form
is close enough.

\begin{figure}[h!]
  \centering
  \includegraphics[scale = 0.5, keepaspectratio=true]{../Figures/dev_np_vs_damage_dev}
  \caption{Nonparametric Model for Tractor Prices: Damage} \label{fig:dev_np_vs_damage_dev}
\end{figure}

 
\clearpage
Finally, consider a set of nonparametric specifications for 
the relationship between cost and damage.
Figure \ref{fig:dev_np_vs_damage_dev_bw} 
overlays other nonparametric estimates with the above in 
Figure \ref{fig:dev_np_vs_damage_dev}.
The points in orange and in magenta represent
alternate models with different degrees of smoothing. 
%
When we estimated probability densities,
we adjusted the bandwidth parameter to fit
with different degrees of smoothness.
The \texttt{loess} method used for the nonparametric method has a span parameter for this function.
The default smoother \texttt{span} (bandwidth parameter) is 0.75.

In the magenta points, with \texttt{span} parameter 0.1, the pattern has more variation in slope but 
closely follows the prediction from the quadratic model. 
The smoother curve in orange 
even more closely represents a quadratic line. 
Again, it appears that the quadratic form
is close enough.
Perhaps the result will be different for other continuous variables in the model.

\begin{figure}[h!]
  \centering
  \includegraphics[scale = 0.5, keepaspectratio=true]{../Figures/dev_np_vs_damage_dev_bw}
  \caption{Nonparametric Model for Tractor Prices: Damage} \label{fig:dev_np_vs_damage_dev_bw}
\end{figure}





\clearpage
\subsection{Nonparametric Specification for Age}

As above, first conduct FWL regressions 
to reduce the problem to two dimensions. 
The models shown in
Table \ref{tab:reg_age_fwl}
illustrate this possibility. 
Model 1 is the same original model in 
Table \ref{tab:reg_sq_cost}. 
Model 2 is a regression omitting the age variable. 
Model 3 is a regression to predict age with the other explanatory variables in Model 2.
Finally, Model 4 shows the coefficient for age
from a regression of the residuals of Model 2
on the residuals from Model 3. 
Notice that these coefficients match those in Model 1. 


\begin{table}
\begin{center}
\begin{tabular}{l c c c c}
\hline
 & Model 1 & Model 2 & Model 3 & Model 4 \\
\hline
(Intercept)      & $1.55532^{***}$  & $1.55349^{***}$  & $0.58223^{***}$  &                \\
                 & $(0.01457)$      & $(0.01455)$      & $(0.09659)$      &                \\
squared\_mileage & $0.00000^{***}$  & $0.00000^{***}$  & $-0.00000^{***}$ &                \\
                 & $(0.00000)$      & $(0.00000)$      & $(0.00000)$      &                \\
make2            & $0.00842$        & $0.00833$        & $0.03083$        &                \\
                 & $(0.00744)$      & $(0.00744)$      & $(0.04943)$      &                \\
make3            & $0.00969$        & $0.00937$        & $0.10224$        &                \\
                 & $(0.00799)$      & $(0.00799)$      & $(0.05306)$      &                \\
make4            & $0.00764$        & $0.00740$        & $0.07386$        &                \\
                 & $(0.00657)$      & $(0.00657)$      & $(0.04363)$      &                \\
make5            & $0.00826$        & $0.00812$        & $0.04149$        &                \\
                 & $(0.00690)$      & $(0.00690)$      & $(0.04584)$      &                \\
make6            & $-0.04529^{***}$ & $-0.04541^{***}$ & $0.03704$        &                \\
                 & $(0.00670)$      & $(0.00670)$      & $(0.04452)$      &                \\
make7            & $-0.03770^{***}$ & $-0.03777^{***}$ & $0.02188$        &                \\
                 & $(0.00707)$      & $(0.00707)$      & $(0.04695)$      &                \\
make8            & $-0.07169^{***}$ & $-0.07128^{***}$ & $-0.12755$       &                \\
                 & $(0.00982)$      & $(0.00982)$      & $(0.06522)$      &                \\
make9            & $-0.06409^{***}$ & $-0.06368^{***}$ & $-0.13242$       &                \\
                 & $(0.01090)$      & $(0.01090)$      & $(0.07241)$      &                \\
damage2          & $-0.11601^{***}$ & $-0.11613^{***}$ & $0.03807$        &                \\
                 & $(0.01183)$      & $(0.01183)$      & $(0.07856)$      &                \\
damage3          & $-0.16372^{***}$ & $-0.16392^{***}$ & $0.06297$        &                \\
                 & $(0.01071)$      & $(0.01071)$      & $(0.07114)$      &                \\
damage4          & $-0.25179^{***}$ & $-0.25199^{***}$ & $0.06250$        &                \\
                 & $(0.01083)$      & $(0.01083)$      & $(0.07192)$      &                \\
damage5          & $-0.23679^{***}$ & $-0.23692^{***}$ & $0.04301$        &                \\
                 & $(0.01104)$      & $(0.01104)$      & $(0.07333)$      &                \\
damage6          & $-0.28664^{***}$ & $-0.28688^{***}$ & $0.07793$        &                \\
                 & $(0.01127)$      & $(0.01127)$      & $(0.07482)$      &                \\
damage7          & $-0.36724^{***}$ & $-0.36759^{***}$ & $0.11218$        &                \\
                 & $(0.01301)$      & $(0.01301)$      & $(0.08637)$      &                \\
damage8          & $-0.41161^{***}$ & $-0.41081^{***}$ & $-0.25535$       &                \\
                 & $(0.02113)$      & $(0.02113)$      & $(0.14031)$      &                \\
damage9          & $-0.39495^{***}$ & $-0.39599^{***}$ & $0.32838$        &                \\
                 & $(0.02595)$      & $(0.02595)$      & $(0.17233)$      &                \\
damage10         & $-0.44599^{***}$ & $-0.44649^{***}$ & $0.15937$        &                \\
                 & $(0.01535)$      & $(0.01535)$      & $(0.10193)$      &                \\
dealer2          & $-0.03791^{***}$ & $-0.03794^{***}$ & $0.01057$        &                \\
                 & $(0.00919)$      & $(0.00919)$      & $(0.06104)$      &                \\
dealer3          & $0.05242^{***}$  & $0.05129^{***}$  & $0.35918^{***}$  &                \\
                 & $(0.00693)$      & $(0.00691)$      & $(0.04590)$      &                \\
dealer4          & $-0.05643^{***}$ & $-0.05600^{***}$ & $-0.13649^{**}$  &                \\
                 & $(0.00637)$      & $(0.00636)$      & $(0.04226)$      &                \\
dealer5          & $0.03791^{***}$  & $0.03747^{***}$  & $0.13987^{*}$    &                \\
                 & $(0.00874)$      & $(0.00874)$      & $(0.05803)$      &                \\
dealer6          & $0.00365$        & $0.00345$        & $0.06191$        &                \\
                 & $(0.00621)$      & $(0.00621)$      & $(0.04126)$      &                \\
dealer7          & $0.03135^{***}$  & $0.03060^{***}$  & $0.23907^{***}$  &                \\
                 & $(0.00746)$      & $(0.00745)$      & $(0.04947)$      &                \\
dealer8          & $-0.05479^{***}$ & $-0.05457^{***}$ & $-0.06745$       &                \\
                 & $(0.00614)$      & $(0.00614)$      & $(0.04076)$      &                \\
dealer9          & $-0.06341^{***}$ & $-0.06304^{***}$ & $-0.11900^{*}$   &                \\
                 & $(0.00902)$      & $(0.00902)$      & $(0.05987)$      &                \\
mileage          & $-0.00000^{***}$ & $-0.00000^{***}$ & $0.00008^{***}$  &                \\
                 & $(0.00000)$      & $(0.00000)$      & $(0.00000)$      &                \\
age              & $-0.00314^{*}$   &                  &                  &                \\
                 & $(0.00135)$      &                  &                  &                \\
type1            & $0.10347^{***}$  & $0.10348^{***}$  & $-0.00034$       &                \\
                 & $(0.00359)$      & $(0.00359)$      & $(0.02387)$      &                \\
age\_resid       &                  &                  &                  & $-0.00314^{*}$ \\
                 &                  &                  &                  & $(0.00135)$    \\
\hline
R$^2$            & $0.23248$        & $0.23215$        & $0.83054$        & $0.00044$      \\
Adj. R$^2$       & $0.23070$        & $0.23042$        & $0.83016$        & $0.00036$      \\
Num. obs.        & $12492$          & $12492$          & $12492$          & $12492$        \\
\hline
\multicolumn{5}{l}{\scriptsize{$^{***}p<0.001$; $^{**}p<0.01$; $^{*}p<0.05$}}
\end{tabular}
\caption{Linear Model for Age: FWL Regressions}
\label{tab:reg_age_fwl}
\end{center}
\end{table}


\pagebreak 
To illustrate the fit of the model, 
Figure \ref{fig:dev_vs_age} shows a scatter plot 
of the residual log prices on age. 
The observations are shown in blue
and the fitted values are shown in red.
The variation in the fitted values results from the 
fact that it is not plotted against the transformed excess age variable used in the regressions.
Still, the linear pattern is apparent
and appears to match the data. 

\begin{figure}[h!]
  \centering
  \includegraphics[scale = 0.5, keepaspectratio=true]{../Figures/dev_vs_age}
  \caption{Linear-Quadratic Model for Tractor Prices} \label{fig:dev_vs_age}
\end{figure}



\pagebreak
As a comparison, Figure \ref{fig:dev_vs_age_dev} 
augments the above by showing the plot against the 
residuals from the regression for age:
the ``excess age'' of a truck compared to what would be 
expected given the other characteristics of the truck. 
Notice that this time the fit follows a straight line,
since we have a single variable with no
quadratic transformation.

\begin{figure}[h!]
  \centering
  \includegraphics[scale = 0.5, keepaspectratio=true]{../Figures/dev_vs_age_dev}
  \caption{Linear-Quadratic Model for trucks costs: Excess Age} \label{fig:dev_vs_age_dev}
\end{figure}

\clearpage
Now consider a nonparametric specification for 
the relationship between cost and age.
Figure \ref{fig:dev_np_vs_age_dev} 
overlays the nonparametric estimate (shown in green) with the above in 
Figure \ref{fig:dev_vs_age_dev}.
The pattern has more variation in slope but 
closely follows the prediction from the linear model. 
Although the nonparametric estimate varies around the linear estimate,
it appears that the linear form
is a close enough approximation without the added complexity.
Next, I will explore the remaining continuous variable.


\begin{figure}[h!]
  \centering
  \includegraphics[scale = 0.5, keepaspectratio=true]{../Figures/dev_np_vs_age_dev}
  \caption{Nonparametric Model for truck cost: Excess Age} \label{fig:dev_np_vs_age_dev}
\end{figure}





\clearpage
\subsection{Nonparametric Specification for dealer}

As above, first conduct FWL regressions 
to reduce the problem to two dimensions. 
The models shown in
Table \ref{tab:reg_eng_fwl}
illustrate this possibility. 
Model 1 is the same original model in 
Table \ref{tab:reg_sq_horse}. 
Model 2 is a regression omitting the age variable. 
Model 3 is a regression to predict dealer with the other explanatory variables in Model 2.
Finally, Model 4 shows the coefficient for engine hours
from a regression of the residuals of Model 2
on the residuals from Model 3. 
Notice that these coefficients match those in Model 1. 


\begin{table}
\begin{center}
\begin{tabular}{l c c c c}
\hline
 & Model 1 & Model 2 & Model 3 & Model 4 \\
\hline
(Intercept)     & $8.76488^{***}$  & $8.77479^{***}$  & $5.04979^{***}$ &             \\
                & $(0.08135)$      & $(0.07858)$      & $(0.18991)$     &             \\
squared\_damage & $0.00459^{*}$    & $0.00460^{*}$    & $0.00438$       &             \\
                & $(0.00214)$      & $(0.00214)$      & $(0.00516)$     &             \\
make            & $-0.00619$       & $-0.00621$       & $-0.01070$      &             \\
                & $(0.00537)$      & $(0.00537)$      & $(0.01297)$     &             \\
damage          & $-0.04677$       & $-0.04684$       & $-0.03332$      &             \\
                & $(0.02543)$      & $(0.02543)$      & $(0.06145)$     &             \\
dealer          & $0.00196$        &                  &                 &             \\
                & $(0.00417)$      &                  &                 &             \\
mileage         & $-0.00001^{***}$ & $-0.00001^{***}$ & $0.00000$       &             \\
                & $(0.00000)$      & $(0.00000)$      & $(0.00000)$     &             \\
age             & $-0.07163^{***}$ & $-0.07170^{***}$ & $-0.03759$      &             \\
                & $(0.01114)$      & $(0.01114)$      & $(0.02692)$     &             \\
type            & $0.04438^{*}$    & $0.04446^{*}$    & $0.04225$       &             \\
                & $(0.02182)$      & $(0.02182)$      & $(0.05274)$     &             \\
type\_resid     &                  &                  &                 & $0.00196$   \\
                &                  &                  &                 & $(0.00417)$ \\
\hline
R$^2$           & $0.19087$        & $0.19085$        & $0.00047$       & $0.00002$   \\
Adj. R$^2$      & $0.19029$        & $0.19036$        & $-0.00014$      & $-0.00008$  \\
Num. obs.       & $9861$           & $9861$           & $9861$          & $9861$      \\
\hline
\multicolumn{5}{l}{\scriptsize{$^{***}p<0.001$; $^{**}p<0.01$; $^{*}p<0.05$}}
\end{tabular}
\caption{Linear Model for type : FWL Regressions}
\label{tab:reg_eng_fwl}
\end{center}
\end{table}


\pagebreak 
To illustrate the fit of the model, 
Figure \ref{fig:dev_vs_eng} shows a scatter plot 
of the residual log cost on dealer. 
The observations are shown in blue
and the fitted values are shown in red.
 

\begin{figure}[h!]
  \centering
  \includegraphics[scale = 0.5, keepaspectratio=true]{../Figures/dev_vs_eng}
  \caption{Linear-Quadratic Model for truck cost} \label{fig:dev_vs_eng}
\end{figure}



\pagebreak
As a comparison, Figure \ref{fig:dev_np_vs_eng_dev} 
augments the above by showing the plot against the 
residuals from the regression for dealer.
% 
I move directly to the nonparametric specification for 
the relationship between cost and dealer.
Figure \ref{fig:dev_np_vs_eng_dev} 
overlays the nonparametric estimate, shown in green. 
The pattern has more variation in slope but 
closely follows the prediction from the linear model. 
Although the nonparametric estimate varies around the linear estimate,
it appears that the linear form
is also a close enough approximation, 
just as was found for the age variable.


\begin{figure}[h!]
  \centering
  \includegraphics[scale = 0.5, keepaspectratio=true]{../Figures/dev_np_vs_eng_dev}
  \caption{Nonparametric Model for Tractor Prices: dealer} \label{fig:dev_np_vs_eng_dev}
\end{figure}

 

\pagebreak
\section{Semiparametric Estimates}


Table \ref{tab:reg_semipar} 
shows the estimates from a set of models. 
Model 1 is the benchmark linear model in 
Table \ref{tab:reg_sq_damage}. 
Model 2 is a semi-parametric model
with a nonparametric fit on horsepower
substituted in for the damage variables.
Models 3 and 4 are semi-parametric models
with nonparametric fits on age and engine hours, respectively.
Model 5 is a maximally semiparametric model, 
with nonparametric fits for all continuous variables. 
For each of the single-variable semiparametric models, 
the coefficients are near one
and the fits are similar to the linear model. 
Even with maximal flexibility, the fit of Model 5
is not much better than the benchmark linear model. 
Across all models, the adjusted $\bar{R}^2$ values are all hovering around 0.80. 
All things considered, these are excellent models
and the linear model is sufficient.


\begin{table}
\begin{center}
\begin{tabular}{l c c c c c}
\hline
 & Model 1 & Model 2 & Model 3 & Model 4 & Model 5 \\
\hline
(Intercept)      & $1.55532^{***}$  & $1.49114^{***}$  & $1.55199^{***}$  & $1.55532^{***}$  & $1.44046^{***}$  \\
                 & $(0.01457)$      & $(0.01318)$      & $(0.01454)$      & $(0.01457)$      & $(0.01343)$      \\
squared\_mileage & $0.00000^{***}$  &                  & $0.00000^{***}$  & $0.00000^{***}$  &                  \\
                 & $(0.00000)$      &                  & $(0.00000)$      & $(0.00000)$      &                  \\
make2            & $0.00842$        & $0.00999$        & $0.00809$        & $0.00842$        & $0.01082$        \\
                 & $(0.00744)$      & $(0.00747)$      & $(0.00744)$      & $(0.00744)$      & $(0.00738)$      \\
make3            & $0.00969$        & $0.00941$        & $0.00930$        & $0.00969$        & $0.00844$        \\
                 & $(0.00799)$      & $(0.00802)$      & $(0.00799)$      & $(0.00799)$      & $(0.00793)$      \\
make4            & $0.00764$        & $0.00724$        & $0.00746$        & $0.00764$        & $0.00394$        \\
                 & $(0.00657)$      & $(0.00660)$      & $(0.00657)$      & $(0.00657)$      & $(0.00653)$      \\
make5            & $0.00826$        & $0.00828$        & $0.00809$        & $0.00826$        & $0.00792$        \\
                 & $(0.00690)$      & $(0.00693)$      & $(0.00690)$      & $(0.00690)$      & $(0.00685)$      \\
make6            & $-0.04529^{***}$ & $-0.04422^{***}$ & $-0.04538^{***}$ & $-0.04529^{***}$ & $-0.04378^{***}$ \\
                 & $(0.00670)$      & $(0.00673)$      & $(0.00670)$      & $(0.00670)$      & $(0.00665)$      \\
make7            & $-0.03770^{***}$ & $-0.03760^{***}$ & $-0.03798^{***}$ & $-0.03770^{***}$ & $-0.03828^{***}$ \\
                 & $(0.00707)$      & $(0.00709)$      & $(0.00707)$      & $(0.00707)$      & $(0.00701)$      \\
make8            & $-0.07169^{***}$ & $-0.06557^{***}$ & $-0.07183^{***}$ & $-0.07169^{***}$ & $-0.06181^{***}$ \\
                 & $(0.00982)$      & $(0.00984)$      & $(0.00982)$      & $(0.00982)$      & $(0.00973)$      \\
make9            & $-0.06409^{***}$ & $-0.05799^{***}$ & $-0.06311^{***}$ & $-0.06409^{***}$ & $-0.05134^{***}$ \\
                 & $(0.01090)$      & $(0.01093)$      & $(0.01090)$      & $(0.01090)$      & $(0.01082)$      \\
damage2          & $-0.11601^{***}$ & $-0.12031^{***}$ & $-0.11567^{***}$ & $-0.11601^{***}$ & $-0.11100^{***}$ \\
                 & $(0.01183)$      & $(0.01186)$      & $(0.01182)$      & $(0.01183)$      & $(0.01174)$      \\
damage3          & $-0.16372^{***}$ & $-0.16932^{***}$ & $-0.16350^{***}$ & $-0.16372^{***}$ & $-0.14894^{***}$ \\
                 & $(0.01071)$      & $(0.01073)$      & $(0.01071)$      & $(0.01071)$      & $(0.01067)$      \\
damage4          & $-0.25179^{***}$ & $-0.25730^{***}$ & $-0.25154^{***}$ & $-0.25179^{***}$ & $-0.23294^{***}$ \\
                 & $(0.01083)$      & $(0.01085)$      & $(0.01082)$      & $(0.01083)$      & $(0.01082)$      \\
damage5          & $-0.23679^{***}$ & $-0.24147^{***}$ & $-0.23643^{***}$ & $-0.23679^{***}$ & $-0.21545^{***}$ \\
                 & $(0.01104)$      & $(0.01107)$      & $(0.01104)$      & $(0.01104)$      & $(0.01105)$      \\
damage6          & $-0.28664^{***}$ & $-0.29135^{***}$ & $-0.28648^{***}$ & $-0.28664^{***}$ & $-0.26132^{***}$ \\
                 & $(0.01127)$      & $(0.01129)$      & $(0.01126)$      & $(0.01127)$      & $(0.01130)$      \\
damage7          & $-0.36724^{***}$ & $-0.37050^{***}$ & $-0.36707^{***}$ & $-0.36724^{***}$ & $-0.33120^{***}$ \\
                 & $(0.01301)$      & $(0.01305)$      & $(0.01300)$      & $(0.01301)$      & $(0.01310)$      \\
damage8          & $-0.41161^{***}$ & $-0.40823^{***}$ & $-0.41083^{***}$ & $-0.41161^{***}$ & $-0.37185^{***}$ \\
                 & $(0.02113)$      & $(0.02120)$      & $(0.02112)$      & $(0.02113)$      & $(0.02106)$      \\
damage9          & $-0.39495^{***}$ & $-0.38629^{***}$ & $-0.39452^{***}$ & $-0.39495^{***}$ & $-0.35503^{***}$ \\
                 & $(0.02595)$      & $(0.02603)$      & $(0.02594)$      & $(0.02595)$      & $(0.02579)$      \\
damage10         & $-0.44599^{***}$ & $-0.40667^{***}$ & $-0.44629^{***}$ & $-0.44599^{***}$ & $-0.37381^{***}$ \\
                 & $(0.01535)$      & $(0.01481)$      & $(0.01534)$      & $(0.01535)$      & $(0.01476)$      \\
dealer2          & $-0.03791^{***}$ & $-0.03683^{***}$ & $-0.03828^{***}$ & $-0.03791^{***}$ & $-0.01240$       \\
                 & $(0.00919)$      & $(0.00923)$      & $(0.00919)$      & $(0.00919)$      & $(0.00923)$      \\
dealer3          & $0.05242^{***}$  & $0.05281^{***}$  & $0.05177^{***}$  & $0.05242^{***}$  & $0.02877^{***}$  \\
                 & $(0.00693)$      & $(0.00695)$      & $(0.00691)$      & $(0.00693)$      & $(0.00701)$      \\
dealer4          & $-0.05643^{***}$ & $-0.05395^{***}$ & $-0.05598^{***}$ & $-0.05643^{***}$ & $-0.01699^{*}$   \\
                 & $(0.00637)$      & $(0.00639)$      & $(0.00636)$      & $(0.00637)$      & $(0.00667)$      \\
dealer5          & $0.03791^{***}$  & $0.03655^{***}$  & $0.03785^{***}$  & $0.03791^{***}$  & $0.00775$        \\
                 & $(0.00874)$      & $(0.00877)$      & $(0.00873)$      & $(0.00874)$      & $(0.00883)$      \\
dealer6          & $0.00365$        & $0.00284$        & $0.00382$        & $0.00365$        & $-0.00494$       \\
                 & $(0.00621)$      & $(0.00623)$      & $(0.00621)$      & $(0.00621)$      & $(0.00618)$      \\
dealer7          & $0.03135^{***}$  & $0.02970^{***}$  & $0.03093^{***}$  & $0.03135^{***}$  & $0.00677$        \\
                 & $(0.00746)$      & $(0.00748)$      & $(0.00745)$      & $(0.00746)$      & $(0.00751)$      \\
dealer8          & $-0.05479^{***}$ & $-0.05116^{***}$ & $-0.05447^{***}$ & $-0.05479^{***}$ & $-0.01498^{*}$   \\
                 & $(0.00614)$      & $(0.00617)$      & $(0.00613)$      & $(0.00614)$      & $(0.00645)$      \\
dealer9          & $-0.06341^{***}$ & $-0.06304^{***}$ & $-0.06284^{***}$ & $-0.06341^{***}$ & $-0.01598$       \\
                 & $(0.00902)$      & $(0.00905)$      & $(0.00901)$      & $(0.00902)$      & $(0.00935)$      \\
mileage          & $-0.00000^{***}$ & $-0.00000$       & $-0.00000^{***}$ & $-0.00000^{***}$ & $-0.00000$       \\
                 & $(0.00000)$      & $(0.00000)$      & $(0.00000)$      & $(0.00000)$      & $(0.00000)$      \\
age              & $-0.00314^{*}$   & $-0.01340^{***}$ &                  & $-0.00314^{*}$   & $-0.00975^{**}$  \\
                 & $(0.00135)$      & $(0.00290)$      &                  & $(0.00135)$      & $(0.00303)$      \\
type1            & $0.10347^{***}$  & $0.09810^{***}$  & $0.10349^{***}$  & $0.10347^{***}$  & $0.09311^{***}$  \\
                 & $(0.00359)$      & $(0.00372)$      & $(0.00359)$      & $(0.00359)$      & $(0.00369)$      \\
mileage\_np      &                  & $0.95522^{***}$  &                  &                  & $1.07634^{***}$  \\
                 &                  & $(0.27848)$      &                  &                  & $(0.27560)$      \\
age\_np          &                  &                  & $1.23265^{***}$  &                  & $0.99555^{*}$    \\
                 &                  &                  & $(0.31198)$      &                  & $(0.40486)$      \\
cost\_np         &                  &                  &                  &                  & $0.82840^{***}$  \\
                 &                  &                  &                  &                  & $(0.04852)$      \\
\hline
R$^2$            & $0.23248$        & $0.22714$        & $0.23311$        & $0.23248$        & $0.24513$        \\
Adj. R$^2$       & $0.23070$        & $0.22534$        & $0.23133$        & $0.23070$        & $0.24326$        \\
Num. obs.        & $12492$          & $12492$          & $12492$          & $12492$          & $12492$          \\
\hline
\multicolumn{6}{l}{\scriptsize{$^{***}p<0.001$; $^{**}p<0.01$; $^{*}p<0.05$}}
\end{tabular}
\caption{Semiparametric Models for ROR}
\label{tab:reg_semipar}
\end{center}
\end{table}






\pagebreak
\section{Generalized Additive Model}

\subsection{Linear Model}

As an example of the output from the GAM specification, 
I first estimated the model with no nonlinear terms, 
which is essentially a linear regression. 

\begin{verbatim}
Family: gaussian 
Link function: identity 

Formula:
log_cost ~ damage + squared_damage + age + make + damage + dealer + 
    mileage + age + type

Parametric coefficients:
                 Estimate Std. Error t value Pr(>|t|)    
(Intercept)     8.765e+00  8.135e-02 107.740  < 2e-16 ***
damage         -4.677e-02  2.543e-02  -1.839   0.0659 .  
squared_damage  4.595e-03  2.135e-03   2.152   0.0314 *  
age            -7.163e-02  1.114e-02  -6.430 1.34e-10 ***
make           -6.194e-03  5.368e-03  -1.154   0.2486    
dealer          1.964e-03  4.168e-03   0.471   0.6376    
mileage        -5.053e-06  1.062e-06  -4.756 2.01e-06 ***
type            4.438e-02  2.182e-02   2.034   0.0420 *  
---
Signif. codes:  0 �***� 0.001 �**� 0.01 �*� 0.05 �.� 0.1 � � 1


R-sq.(adj) =   0.19   Deviance explained = 19.1%
GCV = 1.1444  Scale est. = 1.1435    n = 9861
\end{verbatim}

\pagebreak
\subsection{Semiparametric Model}


Further investigating the results of the full semiparametric specification
in Model 5 of Table \ref{tab:reg_semipar},
I estimated the model with all three continuous variables specified as nonparametric functions. 
The result was that 
almost all the variables---both linear and nonlinear---were 
statistically significant. 
The only exception was a loss in significance of the diesel indicator. 


\begin{verbatim}
Family: gaussian 
Link function: identity 

Formula:
ror ~ s(mileage) + s(age)

Parametric coefficients:
            Estimate Std. Error t value Pr(>|t|)    
(Intercept) 1.191988   0.001692   704.4   <2e-16 ***
---
Signif. codes:  0 �***� 0.001 �**� 0.01 �*� 0.05 �.� 0.1 � � 1

Approximate significance of smooth terms:
             edf Ref.df      F p-value    
s(mileage) 6.059  7.350 20.651  <2e-16 ***
s(age)     8.705  8.946  2.036  0.0233 *  
---
Signif. codes:  0 �***� 0.001 �**� 0.01 �*� 0.05 �.� 0.1 � � 1

R-sq.(adj) =  0.0726   Deviance explained = 7.37%
GCV = 0.035816  Scale est. = 0.03577   n = 12492
\end{verbatim}

On the other hand, 
the adjusted R-squared has not increased very much, 
from 0.799 to 0.819 under this specification, 
which may not justify the added complexity of the model.
Perhaps more importantly, the coefficients on the 
linear terms are very similar across models, 
indicating that the models support similar conclusions relating to any business decision involving
the John Deere premium. 
With this second model, we have even more support for those conclusions
and are certain that the conclusions are not 
coincidental results of the
functional form decisions for previous models.


Perhaps as a middle ground, we can estimate a model with a 
nonparametric specification for the horsepower variable alone, 
since it seems to have a nonlinear relationship with value in either case. 
This retains most of the predictive value of the maximally 
semiparametric model and accommodates the 
nonlinear relationship with value of horsepower. 

\begin{verbatim}
Family: gaussian 
Link function: identity 

Formula:
ror ~ s(mileage) + age + make + damage + dealer + mileage + age + 
    type

Parametric coefficients:
              Estimate Std. Error t value Pr(>|t|)    
(Intercept)  2.338e-01  4.583e-03  51.005  < 2e-16 ***
age         -3.284e-03  1.349e-03  -2.435   0.0149 *  
make2        8.415e-03  7.443e-03   1.131   0.2582    
make3        9.675e-03  7.991e-03   1.211   0.2260    
make4        7.743e-03  6.570e-03   1.179   0.2386    
make5        8.241e-03  6.902e-03   1.194   0.2325    
make6       -4.535e-02  6.703e-03  -6.765 1.39e-11 ***
make7       -3.772e-02  7.069e-03  -5.336 9.68e-08 ***
make8       -7.175e-02  9.829e-03  -7.300 3.05e-13 ***
make9       -6.397e-02  1.091e-02  -5.862 4.68e-09 ***
damage2     -1.155e-01  1.184e-02  -9.756  < 2e-16 ***
damage3     -1.633e-01  1.072e-02 -15.229  < 2e-16 ***
damage4     -2.516e-01  1.083e-02 -23.225  < 2e-16 ***
damage5     -2.367e-01  1.104e-02 -21.433  < 2e-16 ***
damage6     -2.866e-01  1.127e-02 -25.431  < 2e-16 ***
damage7     -3.679e-01  1.301e-02 -28.278  < 2e-16 ***
damage8     -4.135e-01  2.116e-02 -19.540  < 2e-16 ***
damage9     -3.966e-01  2.598e-02 -15.267  < 2e-16 ***
damage10    -4.431e-01  1.540e-02 -28.777  < 2e-16 ***
dealer2     -3.794e-02  9.191e-03  -4.128 3.68e-05 ***
dealer3      5.276e-02  6.929e-03   7.614 2.84e-14 ***
dealer4     -5.639e-02  6.368e-03  -8.855  < 2e-16 ***
dealer5      3.781e-02  8.740e-03   4.326 1.53e-05 ***
dealer6      3.592e-03  6.212e-03   0.578   0.5632    
dealer7      3.139e-02  7.456e-03   4.210 2.57e-05 ***
dealer8     -5.468e-02  6.141e-03  -8.904  < 2e-16 ***
dealer9     -6.329e-02  9.016e-03  -7.020 2.33e-12 ***
mileage      1.348e-05  1.667e-07  80.853  < 2e-16 ***
type1        1.033e-01  3.593e-03  28.756  < 2e-16 ***
---
Signif. codes:  0 �***� 0.001 �**� 0.01 �*� 0.05 �.� 0.1 � � 1

Approximate significance of smooth terms:
             edf Ref.df    F p-value    
s(mileage) 3.652  4.643 2640  <2e-16 ***
---
Signif. codes:  0 �***� 0.001 �**� 0.01 �*� 0.05 �.� 0.1 � � 1

Rank: 37/38
R-sq.(adj) =  0.231   Deviance explained = 23.3%
GCV = 0.029745  Scale est. = 0.029669  n = 12492





%%%%%%%%%%%%%%%%%%%%%%%%%%%%%%%%%%%%%%%%
\end{document}
%%%%%%%%%%%%%%%%%%%%%%%%%%%%%%%%%%%%%%%%
