%%%%%%%%%%%%%%%%%%%%%%%%%%%%%%%%%%%%%%%%%%%%%%%%%%%%%%%%%%%

\documentclass[11pt]{article}
\usepackage{fullpage}
\begin{document}


%%%%%%%%%%%%%%%%%%%%%%%%%%%%%%%%%%%%%%%%%%%%%%%%%%%%%%%%%%%%

{\noindent\bf Spring 2021 \hfill First Last}
\vskip 16pt
\centerline{\bf University of Central Florida}
\centerline{\bf College of Business }
\vskip 16pt
\centerline{\bf QMB 6912}
\centerline{\bf Capstone Project in Business Analytics}
\vskip 10pt
\centerline{\bf Problem Set \#3}
\vskip 32pt
\noindent
% 
\section{Data Description}
% 
By engaging an industry consultant to gather relevant and appropriate 
information, your firm has been able to put together data concerning 248 
different fly-fishing reels, over one-half of which are produced in the 
United States, with the remainder being produced in Asia---either in China 
or Korea.  These data are contained in the file {\tt FlyReels.csv}, which is
available in the {\tt Data} folder.
Each fly-fishing reel in the data set is a row, while the columns correspond 
to the variables whose names and definitions are the following:
\bigskip
\begin{table}[ht]
\centering
\begin{tabular}{ll}
  \hline
    Variable & Definition \\
  \hline

    {\tt Name}        &product name (a string) \\ 
    {\tt Brand}       &brand name (a string) \\ 
    {\tt Weight}      &weight of reel in ounces (a real number) \\ 
    {\tt Diameter}    &diameter of reel in inches (a real number) \\ 
    {\tt Width}       &width of reel in inches (a real number) \\ 
    {\tt Price}       &price of reel in dollars (a real number) \\ 
    {\tt Sealed}      &whether the reel is sealed; {\tt "Yes"} versus
                        {\tt "No"} (a string) \\ 
    {\tt Country}     &country of manufacture, (a string) \\ 
    {\tt Machined}    &whether the reel is machined versus cast;
                        machined={\tt "Yes"}, \\ 
                      &while cast={\tt "No"} (a string) \\ 
  \hline
\end{tabular}
%\caption{Summary of Numeric Variables}
%\label{tab:summary}
\end{table}

\bigskip
\noindent
I have downloaded the file {\tt FlyReels.csv}, 
loaded the data described above into 
\textsf{R}, 
calculated the summary statistics for these data, 
and finally, presented 
these statistics in \LaTeX\ tables.
These operations are all performed by the script 
{\tt FlyReel\_Tables.R} in the {\tt Code} folder. 
The script uses an \textsf{R} package called {\tt xtable} 
to automate the
production of the tables from a data frames in \textsf{R}.)

\medskip
\noindent
I analyze the data in subsets, according to country of manufacture, 
calculating the summary statistics for each subset and present these 
statistics in the \LaTeX\ tables that follow.

\vfill

%%%%%%%%%%%%%%%%%%%%%%%%%%%%%%%%%%%%%%%%%%%%%%%%%%%%%%%%%%%%

\pagebreak
\section{Summary by Country of Manufacture}

Table \ref{tab:summ_by_country} lists summary statistics for numeric variables
in separate columns for subsamples defined by the country of manufacture. 

\input{../Tables/summ_by_country}


\pagebreak
\section{Country of Manufacture by Brand of Fly Reel}

Table \ref{tab:country_by_brand} lists the frequencies of observations of 
each brand of fly reel by country of manufacture. 

\input{../Tables/country_by_brand}

\pagebreak
\section{Reel Design by Brand of Fly Reel}

Table \ref{tab:design_by_brand} lists the frequencies of observations of 
each brand of fly reel across two categorical variables:
whether the reel is sealed
and whether the reel is machined versus cast. 


\input{../Tables/design_by_brand}


%%%%%%%%%%%%%%%%%%%%%%%%%%%%%%%%%%%%%%%%%%%%%%%%%%%%%%%%%%%%


\end{document}

%%%%%%%%%%%%%%%%%%%%%%%%%%%%%%%%%%%%%%%%%%%%%%%%%%%%%%%%%%%%
