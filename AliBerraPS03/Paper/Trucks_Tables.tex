%%%%%%%%%%%%%%%%%%%%%%%%%%%%%%%%%%%%%%%%%%%%%%%%%%%%%%%%%%%

\documentclass[11pt]{article}
\usepackage{fullpage}
\begin{document}


%%%%%%%%%%%%%%%%%%%%%%%%%%%%%%%%%%%%%%%%%%%%%%%%%%%%%%%%%%%%

{\noindent\bf Spring 2021 \hfill Ali Berra}
\vskip 16pt
\centerline{\bf University of Central Florida}
\centerline{\bf College of Business }
\vskip 16pt
\centerline{\bf QMB 6912}
\centerline{\bf Capstone Project in Business Analytics}
\vskip 10pt
\centerline{\bf Problem Set \#3}
\vskip 32pt
\noindent
% 
\section{Data Description}
% 
Several Dealerships has gather releveant and appropriate information, and organized a dataset concerning 9,861 sales invlolving a trade-in of truck at nine dealerships.  These data are contained in the file {\tt UsedTrucks.dat}, which is
available in the {\tt Data} folder.
Each Truck sale in the data set is a row, while the columns correspond 
to the variables whose names and definitions are the following:
\bigskip
\begin{table}[ht]
\centering
\begin{tabular}{ll}
  \hline
    Variable & Definition \\
  \hline

    {\tt type}        &sale type \\ 
    {\tt pauc}       &price when sold at auction \\ 
    {\tt pret}      &price when sold retail \\ 
    {\tt mileage}    &odometer \\ 
    {\tt make}       &make of vehicule \\ 
    {\tt year}       &model year of vehicle \\ 
    {\tt damage}      &an index of damage to vehicle, 1 little damage, 10 a lot\\
    {\tt dealer}     &dealer id \\ 
    {\tt ror}    &rate-of-return \\ 
    {\tt ror}             &net amount given to trade-in \\ 
  \hline
\end{tabular}
%\caption{Summary of Numeric Variables}
%\label{tab:summary}
\end{table}

\bigskip
\noindent
I have downloaded the file {\tt UsedTrucks.dat}, 
loaded the data described above into 
\textsf{R}, 
calculated the summary statistics for these data, 
and finally, presented 
these statistics in \LaTeX\ tables.
These operations are all performed by the script 
{\tt UsedTrucks\_Tables.R} in the {\tt Code} folder. 
The script uses an \textsf{R} package called {\tt xtable} 
to automate the
production of the tables from a data frames in \textsf{R}.)

\medskip
\noindent
I analyze the data in subsets, according to Type , 
calculating the summary statistics for each subset and present these 
statistics in the \LaTeX\ tables that follow.

\vfill

%%%%%%%%%%%%%%%%%%%%%%%%%%%%%%%%%%%%%%%%%%%%%%%%%%%%%%%%%%%%

\pagebreak
\section{Summary by Sale Type}

Table2 lists summary statistics for numeric variables
in separate columns for subsamples defined by Type ( o for Trucks sold in auction and 1 for Trucks sold in retail). 

% latex table generated in R 4.0.2 by xtable 1.8-4 package
% Thu Feb 10 17:14:41 2022
\begin{table}[ht]
\centering
\begin{tabular}{rll}
  \hline
 & 0 & 1 \\ 
  \hline
Min. ror & 1.0493 & 1.0999 \\ 
  Mean ror & 1.0497 & 1.1000 \\ 
  Max. ror & 1.0493 & 1.0999 \\ 
   \hline
\end{tabular}
\caption{Summary by type } 
\label{tab:summ_by_type}
\end{table}




\section{By Dealer}

Table \ref{tab: Dealer} lists the frequencies of observations of 
each Dealer . 

\input{../Tables/By_Dealer}

\pagebreak
\section{Sales by Auto Maker ( Ford and Chevrolet)}

Table3


% latex table generated in R 4.0.2 by xtable 1.8-4 package
% Thu Feb 10 16:49:57 2022
\begin{table}[ht]
\centering
\begin{tabular}{rrrrr}
  \hline
 & Ford & Ford & Chevrolet & total \\ 
  \hline
1 & 156 & 171 & 136 & 1432 \\ 
  2 & 162 & 144 & 129 & 1354 \\ 
  3 & 144 & 165 & 150 & 1436 \\ 
  4 & 153 & 166 & 156 & 1380 \\ 
  5 & 163 & 147 & 161 & 1403 \\ 
  6 & 160 & 156 & 158 & 1427 \\ 
  7 & 169 & 165 & 157 & 1429 \\ 
  Totals & 1107 & 1114 & 1047 & 9861 \\ 
   \hline
\end{tabular}
\caption{Brand of Used Trucks} 
\label{tab:Truck_brand}
\end{table}



%%%%%%%%%%%%%%%%%%%%%%%%%%%%%%%%%%%%%%%%%%%%%%%%%%%%%%%%%%%%


\end{document}

%%%%%%%%%%%%%%%%%%%%%%%%%%%%%%%%%%%%%%%%%%%%%%%%%%%%%%%%%%%%
